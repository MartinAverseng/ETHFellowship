\documentclass[]{article}
\usepackage[a4paper, total={6in, 8in}]{geometry}
\usepackage{amsmath}
\usepackage{amssymb}
\usepackage{color}
\usepackage{setspace}
\usepackage{cite}
\date{}
%opening
\title{A02\_Project Description – Application to the ETH Zurich Postdoctoral Fellowship programme}
\let\OLDthebibliography\thebibliography
\renewcommand\thebibliography[1]{
	\OLDthebibliography{#1}
	\setlength{\parskip}{0pt}
	\setlength{\itemsep}{0pt plus 0.5ex}
}
\begin{document}

\maketitle


\section*{2.1 Summary}

Key objectives of the proposed project(s) of the ETH Fellow, methods to be employed, milestones and expected output; should be comprehensible for the educated layman. Please do not exceed the frame below. Please do not use references or footnotes in the summary. \\



\noindent\fbox{%
	\parbox{\textwidth}{%
		Integral equations are a class of mathematical methods to recast time-harmonic scattering problems into functional equations satisfied by surface densities defined on the scattering obstacle. Various techniques are known to approximate the solution of those equations numerically. In many cases, the system matrix of the discrete problem is ill-conditioned. Unfavorable cases typically involve scatterers with singularities like edges or conical points, and very low / very high frequency regimen. 
		In this research project, we aim at designing new numerical schemes for integral equations that are well conditioned, able to deal with complex geometry of the scatterers and robust in the low and high frequency limits. Mathematical analysis will be used to prove rigorously the favorable properties of those new methods. Moreover, the methods will be systematically implemented and tested numerically to illustrate the theoretical findings. 
		The first method we will design is a combined-field method for multi-screens. Multi-screens are among the most general and complex types of scatterers that are currently tackled by the litterature. Combined field methods are a known class of methods for integral equations that are very efficient for simple scatterers but have failed for simple screens. The recent formalism of multi-screens seems perfectly fitted for the application of this idea and our preliminary numerical tests indicate that the idea is extremely promising. 
		The second method is a conformal mapping preconditioner for integral equations on polygons. This is a generalization of the applicant’s thesis work. The heart of the idea is a simple identity relating the conformal mapping of the polygon into a circle to the Dirichlet-Neumann operator. This new understanding of the Dirichlet-Neumann operator translates into a new preconditioner and hints at a new kind of theoretical tools to analyze integral equations on polygons. Very promising preliminary analysis and numerical tests are readily available. 
		Long term, we have in mind the following objectives. Pursue generalizations of the previous ideas to increasing levels of complexity, to fill the gap between theory and practical applications. Especially adapting all those methods to the electromagnetic integral equations in dimensions 3.
	}%
}

\section*{2.2 Project details}
(rationale for the proposed work, goals, experimental approach and/or methods, milestones, expected output, specific role of the ETH Fellow, timetable; not to exceed 5 pages)

\subsection*{Introduction}


\subsubsection*{Scattering problems}	

Assume that an incident harmonic wave of frequency $k$ is sent to an obstacle, or a material heterogeneity (an airplane in the sky, a crack in a material, a tumor in the body...). The presence of the obstacle generates a scattered field propagating away from this heterogeneity. The shape and intensity of this scattered field depends essentially on the location, geometry and material properties of the scatterer. In many contexts it is fundamental to predict by numerical computation, with high fidelity and in a short time, the scattered field from the \textit{a priori} knowledge of the scattering object and incident wave. 

Scattering problems arise for several different types of waves, by increasing level of complexity:
\begin{itemize}\itemsep-.2em 
	\item[-] Acoustics, sound waves in dimensions 2 and 3.
	\item[-] Electromagnetism, Maxwell equations
	\item[-] Linear elasticity. 
\end{itemize} 
Usually, practical applications involve the last two kinds of waves, at high frequency, and obstacles with complex geometries.  

\subsubsection*{Integral equations}

Among the standard mathematical tools most adapted to this task are the formalism of integral equations and the Galerkin boundary element method (BEM). Reference monographs include e.g. \cite{mclean2000strongly,sauter2010boundary}). The discretization of integral equations by BEM leads to dense linear systems, which, in real-life applications, often have more than $10^6$ unknowns. This makes it prohibitive to solve the linear system by a direct method such as LU factorization, and instead, it is classical to use iterative methods such as GMRES \cite{saad1986gmres}.

For the so-called first-kind integral equations, powerful mesh-related convergence results are well-known, but the linear systems are notoriously ill-conditioned, causing the iterative solvers to progress very slowly. Active research has been devoted to this issue in the past 30 years. On the one hand, some teams have targeted the iteration (matrix-vector product) time by developing compression and acceleration methods e.g. Fast Multipole Methods \cite{greengard1987fast} and H-matrices \cite{hackbusch1999sparse}. On the other hand, other teams have focused on finding suitable preconditioners in order to reduce the number of GMRES iterations. In this branch, \cite{steinbach1998construction} was a pioneer work. 

Second-kind integral equations on the other hand are known to enjoy better stability with respect to discretization parameters, and for this reason, in many instances, researchers have tried to avoid the use of first-kind integral equations despite their simplicity. Nevertheless, the approximation properties of the second-kind formulations are less clear, and they degenerate for complex scatterers such as screens. 


\subsection*{Recent advances and open problems}

Let us survey some recent advances in the field of integral equations, and related open problems.

\subsubsection*{Scattering by complex obstacles}

The obstacles become complex when they are not smooth domains or Lipschitz domains, a class of scatterers that has become very classical with \cite{mclean2000strongly}. Problems arise when singularities are present (edges, conical points). There is a particular attention devoted to screens and cracks, which are the first example of non-Lipschitz domains \cite{bruno2013high,gimperlein2019optimal,hiptmair2018closed,hiptmair2019preconditioning,hiptmair2020optimal,hiptmair2020compact,ramaciotti2017some}. In this case, the formulation of well-posed integral equations in a proper mathematical framework is well-understood since \cite{stephan1984augmented,wendland1990hypersingular}, and the recent advances lie in the proposition of more efficient numerical methods, including high-order approximation schemes and provably ``optimal preconditioners". Additionally, some more challenging geometries are considered, including multi-screens \cite{claeys2013integral,claeys2020quotient} and fractal screens \cite{chandlerWilde2017sobolev}. In those cases, the achievement was to obtain well-posed integral equations and study their theoretical properties. For those complex geometries, many natural questions remain open, e.g. convergence of the numerical methods, singularity analysis and efficient preconditioning. 

\subsubsection*{Dependance in the frequency}

Although optimal preconditioners are by now available for scattering problems by Lipschitz domains and screens, \cite{steinbach1998construction,hiptmair2006operator}, it is a difficult matter to analyze the preconditioning performance in terms of the frequency number. The English school has been concerned with theoretical asymptotic analysis of the condition number of the classical integral equations without preconditioners \cite{chandler2020high} and the use of high-frequency solutions as trial functions in the Galerkin setting to improve the convergence rate \cite{chandlerWilde2015high}. Besides the popular Calder\'{o}n preconditioners \cite{christiansen2000preconditionneurs,christiansen2002preconditioner}, which have recently been adapted to the screen context \cite{bruno2012second},  ``analytic preconditioners" based on pseudo-differential calculus which exhibit good frequency robustness \cite{antoine2007generalized} have been proposed, which entertain a close link with regularized combined-field integral equations \cite{buffa2005regularized}. Although the frequency robustness properties are clear in practice and extensively surveyed e.g. \cite{boubendir2014well}, there is no mathematical theory adapted to the study of this asymptotic robustness as to now. There is a feeling in the community that pseudo-differential analysis, which is often involved in constructing those robust preconditioners, should also play a role in providing quantitative estimates for this frequency dependence buch such desirable analysis has not been achieved yet. Some recent activity around semi-classical analysis seems to go in the right direction \cite{galkowski2019wavenumber}. I will attend a conference on this topic in June 2021 at the university of Bath. 


\subsection*{Own work}

During my PhD thesis, I have contributed to the research on compression and acceleration methods for the linear systems coming from boundary element methods \cite{averseng2017fast}. 
On the other hand, I worked on new high-frequency preconditioners for acoustics integral equations on screens in dimension 2. The first publication on this topic \cite{alouges2019new} presents the analytical framework and numerical testing. The new preconditioners are analog to those of \cite{antoine2007generalized} and present similar frequency robustness, but this time on a screen context. A significant advantage of the method is that the preconditioners obtained are ``quasi-local" in the sense that their use involve only sparse matrix, as opposed to Calder\'{o}n preconditioners which are dense and require an additional compression steps. On the other hand, \cite{averseng2019pseudo} is a more in-depth theoretical paper in which I develop a new type of pseudo-differential calculus on open curves that was used to design the preconditioners introduced in the first paper. This pseudo-differential calculus features the possibility of computing symbols of the layer potentials up to arbitrary orders and is probably a first step in assessing the frequency robustness of this approach. 



\subsection*{Goals and research plan}


\subsubsection*{Frequency-robust preconditioners on 3D screens}

Although the numerical method published in \cite{alouges2019new} is only adapted to 2-dimensional contexts, generalizations to 3-dimensions seem possible and have been attempted in the last chapter of my thesis, with promising results. I will continue my research on this topic in order to make a publication for scalar scattering problems by screens in 3D. In turn, a natural extension to electromagnetic scattering will be possible, as a natural combination of my work and the host laboratory \cite{hiptmair2019preconditioning}. Another possible direction is to extend the pseudo-differential calculus of \cite{averseng2019pseudo} to screens in 3D. 

\subsubsection*{Conformal preconditioners}

Another very natural generalization of my thesis work is the treatment of polygonal scatterers in 2D. We have recently found a mathematical identity that draws a close link between preconditioners and conformal mappings, hinting at completely new and powerful preconditioners adapted to corner singularities. A middle-term objective of this research project will be to analyze rigorously this idea and implement it efficiently. The expected output is a rather complete framework to solve scattering problems by polygons in 2D at any frequency, which would be a significant achievement in the domain. In addition, it might be possible to transport the pseudo-differential to the polygon via the conformal mapping, which would be a very interesting novelty. 

This work would also open some possible breakthrough for 3D singularities such as wedges, which are omnipresent in practical applications. Exploring this idea is a long-term objective of this project.


\subsubsection*{Integral equations on multi-screens}

Another promising idea is the possibility of formulating ``combined-field" integral equations for multi-screens. Combined field formulations are well-known for smooth scatterers and have been introduced more than 50 years ago \cite{brakhage1965dirichletsehe} as a way to guarantee the well-posedness of the equation at any wave frequency. For screens, combined field integral equations have been abandoned, because they degenerate when the obstacle no longer has interior. 
However, we have recently discovered that the formalism of multi-screens restores a ``virtual" interior to the obstacle, and a combined-field equation can be formulated mathematically without difficulty. Preliminary numerical simulations have shown the power of this idea and it will be an immediate objective of this project to implement this idea and analyze it rigorously. This would lead without doubts to an important publication in our field. 

Lastly, after his thesis, I have done a 6 month post-doctorate with Xavier Claeys, a former student of Pr. Hiptmair, on the topic of preconditioning integral equations on multi-screens using an additive Schwarz approach (see e.g. \cite{marchand2019two}). The resulting method is expected to be very robust and versatile. This collaboration will be continued during the post-doc and eventually lead to a publication. 


\subsection*{Timetable}

A timetable of the research project can be proposed as follows:
\begin{itemize}\itemsep-0.2em 
	\item[-] Submission of the paper on preconditioners for 3D screens for scalar waves by December 2020.
	\item[-] Submission of the paper on preconditioners for 3D screens for electromagnetic waves by March 2021.
	\item[-] Submission of the conformal preconditioners paper by July 2021.
	\item[-] In parallel of those, work on the combined-field multi-screens formulations, expected publication by end of 2021. 
	\item[-] Also in parallel, continued work on the additive Schwarz preconditioners and publication before end of 2021.
\end{itemize}
\begin{small}
	\begin{thebibliography}{0}
		
		\bibitem{alouges2019new}
		F.~Alouges and M.~Averseng.
		\newblock New preconditioners for Laplace and Helmholtz integral equations on open curves: Analytical framework and Numerical results. 
		\newblock Accepted for publication in {\em Numerische Mathematik}, 2020.
		
		\bibitem{antoine2007generalized}
		X.~Antoine and M.~Darbas.
		\newblock Generalized combined field integral equations for the iterative
		solution of the three-dimensional helmholtz equation.
		\newblock {\em ESAIM: Mathematical Modelling and Numerical Analysis}
		41(1):147--167, 2007.
		
		\bibitem{averseng2017fast}
		M.~Averseng.
		\newblock Fast discrete convolution in $\mathbb {R}^{2} $ with radial kernels using non-uniform fast Fourier transform with nonequispaced frequencies. 
		\newblock {\em Numerical Algorithms} 83(1):33--56, 2020.
		
		\bibitem{averseng2019pseudo}
		M.~Averseng. 
		\newblock Pseudo-differential analysis of the Helmholtz layer potentials on open curves. 
		\newblock {\em arXiv} preprint:1905.13604.
		
		\bibitem{boubendir2014well}
		Y.~Boubendir and C.~Turc.
		\newblock Well-conditioned boundary integral equation formulations for the solution of high-frequency electromagnetic scattering problems.
		\newblock {\em Computers and Mathematics with Applications} 67(10) 1772--1805, 2014.
		
		\bibitem{brakhage1965dirichletsehe}
		A.~Brakhage, P.~Werner.
		\newblock Uber das dirichletsehe aussenraumproblem fur die Helmholtzsche Schwingungsgleichung
		\newblock {\em Archive fur Mathematik} 16:325–329, 1965.
		
		\bibitem{bruno2012second}
		O.~P.~Bruno and S.~K.~Lintner.
		\newblock Second-kind integral solvers for TE and TM problems of diffraction by open arcs.
		\newblock {\em Radio Science} 47(6):1--13, 2012.
		
		\bibitem{bruno2013high}
		O.~P.~Bruno and S.~K.~Lintner.
		A high-order integral solver for scalar problems of diffraction by screens and apertures in three-dimensional space.
		\newblock {\em Journal of Computational Physics} 252:250--274, 2013.
		
		\bibitem{buffa2005regularized}
		A.~Buffa and R.~Hiptmair.
		\newblock Regularized combined field integral equations.
		\newblock {\em Numerische Mathematik} 100(1):1--19, 2005.	
		
		\bibitem{christiansen2000preconditionneurs}
		S.~H.~Christiansen and J.~C.~Nédélec.
		\newblock Des préconditionneurs pour la résolution numérique des équations intégrales de frontière de l'acoustique.
		\newblock {\em Comptes Rendus de l'Académie des Sciences-Series I-Mathematics}  330(7):617--622, 2000.
		
		\bibitem{christiansen2002preconditioner}
		S.~H.~Christiansen and J.~C.~Nédélec.
		\newblock A preconditioner for the electric field integral equation based on Calderon formulas.
		\newblock {\em SIAM journal on numerical analysis}  40(3):1100--1135, 2002.
		
		\bibitem{chandlerWilde2015high}
		S.~N.~Chandler-Wilde, S.~N.~Hewett, D.~P.~ Langdon and A.~Twigger.
		\newblock A high frequency boundary element method for scattering by a class of nonconvex obstacles.
		\newblock {\em Numerische Mathematik} 129(4):647--689, 2015.
		
		\bibitem{chandlerWilde2017sobolev}
		S.~N.~ Chandler-Wilde, D.~P.~ Hewett and A.~Moiola.
		\newblock Sobolev Spaces on Non-Lipschitz Subsets of ${\mathbb {R}}^n$ with Application to Boundary Integral Equations on Fractal Screens. 
		\newblock {\em Integral Equations and Operator Theory} 87(2):179--224, 2017.
		
		\bibitem{chandler2020high}
		S.~N.~ Chandler-Wilde, E.~Spence, A.~Gibbs and V.~P.~ Smyshlyaev.
		\newblock High-frequency bounds for the Helmholtz equation under parabolic trapping and applications in numerical analysis.
		\newblock {\em SIAM Journal on Mathematical Analysis}52(1):845--893, 2020.	
		
		\bibitem{claeys2013integral}
		X.~Claeys and R.~Hiptmair.
		\newblock Integral equations on multi-screens
		\newblock {\em Integral equations and operator theory} 77(2): 167--197, 2013.
		
		\bibitem{claeys2020quotient}
		X.~Claeys, L.~Giacomel, R. Hiptmair and C.~Urzua-Torres.
		\newblock Quotient-Space Boundary Element Methods for Scattering at Complex Screens.
		\newblock {\em SAM Research Report} 2020, ETH Zurich.
		
		\bibitem{galkowski2019wavenumber}
		J.~Galkowski and E.~A.~Spence
		\newblock Wavenumber-explicit regularity estimates on the acoustic single-and double-layer operators
		\newblock {\em Integral Equations and Operator Theory}91(6), 2019.
		
		\bibitem{gimperlein2019optimal}
		H.~Gimperlein, J.~Stocek and C. Urzua-Torres.
		\newblock Optimal operator preconditioning for pseudodifferential boundary problems.
		\newblock {\em arXiv} preprint 1905.03846, 2019.
		
		\bibitem{greengard1987fast}
		Greengard, L., Rokhlin, V.:
		\newblock A fast algorithm for particle simulations
		\newblock {\em Journal of computational physics} 73(2):325--348, 1987.
		
		\bibitem{hackbusch1999sparse}
		W.~Hackbusch.
		\newblock A sparse matrix arithmetic based on H-matrices. Part I: Introduction to H-matrices.
		\newblock {\em Computing} 62(2):89--108, 1999.
		
		\bibitem{hiptmair2006operator}
		R.~Hiptmair.
		\newblock Operator preconditioning.
		\newblock {\em Computers and mathematics with Applications} 52(5):699--706, 2006.
		
		
		\bibitem{hiptmair2018closed}
		R.~Hiptmair, C.~Jerez-Hanckes and C. Urzua-Torres.
		\newblock Closed-form inverses of the weakly singular and hypersingular operators on disks.
		\newblock {\em Integral Equations and Operator Theory} 90(1), 2018.
		
		
		\bibitem{hiptmair2019preconditioning}
		R.~Hiptmair and C.~Urzua Torres.
		Preconditioning the EFIE on Screens
		\newblock {\em SAM Research Report} 2019, ETH Zurich.
		
		
		\bibitem{hiptmair2020optimal}
		R.~Hiptmair, C.~Jerez-Hanckes and C.~Urzua-Torres.
		\newblock Optimal operator preconditioning for Galerkin boundary element methods on 3D screens. 
		\newblock {\em SIAM J. Numer. Anal.} 58(1):834--857, 2020.
		
		\bibitem{hiptmair2020compact}
		R.~Hiptmair and C.~Urzua-Torres.
		\newblock Compact equivalent inverse of the {E}lectric {F}ield
		{I}ntegral {O}perator on screens.
		\newblock {\em Integral Equations and Operator Theory} 92(1), 2020.

		\bibitem{marchand2019two}
		P.~Marchand, P.~Jolivet, P.~H.~Tournier, X.~Claeys and F.~Nataf
		\newblock Two-level preconditioning for h-version boundary element approximation of hypersingular operator with GenEO.
		\newblock {Archives ouvertes} preprint hal-0218871, 2019.

		
		\bibitem{mclean2000strongly}
		W.~C.~H. McLean.
		\newblock {\em Strongly elliptic systems and boundary integral equations}.
		\newblock Cambridge university press, 2000.
		
		\bibitem{ramaciotti2017some}
		J.~C.~Nédélec and P.~Ramaciotti.
		\newblock About some boundary integral operators on the unit disk related to the Laplace equation
		\newblock {\em SIAM Journal on Numerical Analysis} 55(4):1892--1914, 2017.
		
		
		\bibitem{saad1986gmres}
		Y.~Saad and M.~H. Schultz.
		\newblock GMRES: A generalized minimal residual algorithm for solving
		nonsymmetric linear systems.
		\newblock {\em SIAM Journal on scientific and statistical computing},
		7(3):856--869, 1986.
		
		\bibitem{sauter2010boundary}
		S.~Sauter and C.~Schwab.
		\newblock {\em Boundary Element Methods}
		\newblock Springer 2010.
		
		\bibitem{stephan1984augmented}
		E.~P.~Stephan and W.~L.~Wendland.
		\newblock An augmented Galerkin procedure for the boundary integral method applied to two-dimensional screen and crack problems
		\newblock {\em Applicable Analysis} 18(3):183--219, 1984.
		
		
		\bibitem{steinbach1998construction}
		O.~Steinbach and W.~L. Wendland.
		\newblock The construction of some efficient preconditioners in the boundary
		element method.
		\newblock {\em Advances in Computational Mathematics} 9(1-2):191--216, 1998.
		
		\bibitem{wendland1990hypersingular}
		\newblock A hypersingular boundary integral method for two-dimensional screen and crack problems.
		\newblock {\em Archive for rational mechanics and analysis} 112(4):363--390, 1990.
		

		
		
	\end{thebibliography}
\end{small}

\section*{2.3 Available resources of the ETH host professor for the realization of the project}
(Equipment, staff, knowledge, experience)\\

The proposed project is of a theoretical and computational nature without an experimental component and does not require any special equipment of facilities, beyond standard resources like office space and IT infrastructure. Researchers a the Seminar of Applied Mathematics have access to D-MATH compute servers and ETH's Euler and Leonhard Linux clusters, which can be used for large-scale computations to be carried out as part of the project.

\section*{2.4 Significance of the project for the ETH host professor} 

Prof. R. Hiptmair's research has had a focus on boundary integral equations, boundary element methods, and preconditioning techniques for many years. In particular he has worked on operator preconditioning, screen problems, and novel BIE approaches to electromagnetic scattering. The natural continuation of this work are investigations into operator preconditioning for complex screens and the developmen of preconditioning strategies that are robust for high and low (in the case of electromagnetics) frequencies. Thus Prof. Hiptmair's research interests are perfectly aligned with the goals of the fellowship application.


\section*{2.5 Planned use of requested budget for research costs }
(including travel costs, conference attendances, consumables)\\

Beside kCHF 4 to be spent for travel, I am planning to use the remaining kCHF 8 of the standard budget of kCHF 12 to pay the salary of 1-2 student research assistants, who are to help me with coding jobs. I hope to be able to hire students after they have done term or thesis projects with me. Thus they can make a substantial contribution to software development in the project.

\section*{2.6 Proposed teaching duties of ETH Fellowship candidate}
(including an estimate of the average load)\\

Upon arriving at ETH Zurich I would like to teach an MSc-level course on "Integral
Equation Methods for Scattering" beside a Seminar on "Operator Preconditioning". I hope to attract several undergraduate students interested in doing MSc and BSc thesis or term projects under my supervision, thus building a small informal group. I expect a ~30\% workload due to teaching proper (~15\%) and supervision (~15\%), though the latter will be closely connected to my research.

\section*{2.7 Other requests for funding, especially applications submitted to other fellowship programs}
	(Have you applied to other funding bodies with a similar or the same research project? Please note that you should also inform us if you obtain or request additional funds from other funding bodies during the evaluation period of your application.\\
	
I have not submitted any other applications.

\section*{2.8 List of relevant publications by the ETH Fellow candidate and the ETH host professor}

\begin{thebibliography}{0}
	% Numbers 2 to avoid bibliographic collisions.
	\bibitem{alouges2019new2}
	F.~Alouges and M.~Averseng.
	\newblock New preconditioners for Laplace and Helmholtz integral equations on open curves: Analytical framework and Numerical results. 
	\newblock Accepted for publication in {\em Numerische Mathematik}, 2020.
	
	\bibitem{averseng2017fast2}
	M.~Averseng.
	\newblock Fast discrete convolution in $\mathbb {R}^{2} $ with radial kernels using non-uniform fast Fourier transform with nonequispaced frequencies. 
	\newblock {\em Numerical Algorithms} 83(1):33--56, 2020.
	
	\bibitem{averseng2019pseudo2}
	M.~Averseng. 
	\newblock Pseudo-differential analysis of the Helmholtz layer potentials on open curves. 
	\newblock {\em arXiv} preprint:1905.13604.
	
	\bibitem{CLH17}
	X.~Claeys and R.~Hiptmair.
	\newblock First-kind boundary integral equations for the {H}odge-{H}elmholtz operator.
	\newblock {\em SIAM Journal on Mathematical Analysis} 51(1):197--227, 2019.
	
	\bibitem{CLH18t}
	X.~Claeys and R.~Hiptmair.
	\newblock First-Kind Galerkin Boundary Element Methods for the {H}odge-{L}aplacian in Three Dimensions.
	\newblock {\em SAM report} 2018(35), ETH Zurich.
	
	\bibitem{claeys2020quotient2}
	X.~Claeys, L.~Giacomel, R. Hiptmair and C.~Urzua-Torres.
	\newblock Quotient-Space Boundary Element Methods for Scattering at Complex Screens.
	\newblock {\em SAM Research Report} 2020, ETH Zurich.
	
	
	\bibitem{hiptmair2006operator2}
	R.~Hiptmair.
	\newblock Operator preconditioning.
	\newblock {\em Computers and mathematics with Applications} 52(5):699--706, 2006.
	
	
	\bibitem{hiptmair2018closed2}
	R.~Hiptmair, C.~Jerez-Hanckes and C. Urzua-Torres.
	\newblock Closed-form inverses of the weakly singular and hypersingular operators on disks.
	\newblock {\em Integral Equations and Operator Theory} 90(1), 2018.
	
	\bibitem{hiptmair2019preconditioning2}
	R.~Hiptmair and C.~Urzua Torres.
	Preconditioning the EFIE on Screens
	\newblock {\em SAM Research Report} 2019, ETH Zurich.
	

	\bibitem{hiptmair2020optimal2}
	R.~Hiptmair, C.~Jerez-Hanckes and C.~Urzua-Torres.
	\newblock Optimal operator preconditioning for Galerkin boundary element methods on 3D screens. 
	\newblock {\em SIAM J. Numer. Anal.} 58(1):834--857, 2020.
	
	\bibitem{hiptmair2020compact2}
	R.~Hiptmair and C.~Urzua-Torres.
	\newblock Compact equivalent inverse of the {E}lectric {F}ield
	{I}ntegral {O}perator on screens.
	\newblock {\em Integral Equations and Operator Theory} 92(1), 2020.
	
\end{thebibliography}


\section*{2.9 Relevant publications of other authors}


\begin{thebibliography}{0}
	\bibitem{antoine2007generalized2}
	X.~Antoine and M.~Darbas.
	\newblock Generalized combined field integral equations for the iterative
	solution of the three-dimensional helmholtz equation.
	\newblock {\em ESAIM: Mathematical Modelling and Numerical Analysis}
	41(1):147--167, 2007.
	
	\bibitem{bruno2012second2}
	O.~P.~Bruno and S.~K.~Lintner.
	\newblock Second-kind integral solvers for TE and TM problems of diffraction by open arcs.
	\newblock {\em Radio Science} 47(6):1--13, 2012.
	
	\bibitem{galkowski2019wavenumber2}
	J.~Galkowski and E.~A.~Spence
	\newblock Wavenumber-explicit regularity estimates on the acoustic single-and double-layer operators
	\newblock {\em Integral Equations and Operator Theory}91(6), 2019.
	
	\bibitem{gimperlein2019optimal2}
	H.~Gimperlein, J.~Stocek and C. Urzua-Torres.
	\newblock Optimal operator preconditioning for pseudodifferential boundary problems.
	\newblock {\em arXiv} preprint 1905.03846, 2019.
	
	\bibitem{steinbach1998construction2}
	O.~Steinbach and W.~L. Wendland.
	\newblock The construction of some efficient preconditioners in the boundary
	element method.
	\newblock {\em Advances in Computational Mathematics} 9(1-2):191--216, 1998.
	
	
\end{thebibliography}


\end{document}
