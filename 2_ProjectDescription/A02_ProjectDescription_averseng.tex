\documentclass[]{report}
\usepackage[a4paper, total={6in, 8in}]{geometry}
\usepackage{amsmath}
\usepackage{amssymb}
\usepackage{color}
\usepackage{setspace}
\usepackage{cite}
\usepackage{hyperref}
\date{}
%opening
\title{A02\_Project Description – Application to the ETH Zurich Postdoctoral Fellowship programme}
\let\OLDthebibliography\thebibliography
\renewcommand\thebibliography[1]{
  \OLDthebibliography{#1}
  \setlength{\parskip}{0pt}
  \setlength{\itemsep}{0pt plus 0.5ex}
}

\newcommand{\rh}[1]{{\color{magenta}#1}}
\newcommand{\drop}[1]{{\color{blue}#1}}

\begin{document}

\maketitle

\setcounter{section}{0}
\setcounter{chapter}{2}

\section{Summary}

\drop{ Key objectives of the proposed project(s) of the ETH Fellow, methods to be
  employed, milestones and expected output; should be comprehensible for the educated
  layman. Please do not exceed the frame below. Please do not use references or footnotes
  in the summary.}

\noindent\fbox{%
  \parbox{\textwidth}{%
    The project is concerned with the numerical simulation of acoustic and electromagnetic
    scattering of time-harmonic waves, which is a core task in computational science and
    engineering and enabling technology for optimization and imaging. A popular class of
    numerical methods relies on integral equations, which recast time-harmonic scattering
    problems into functional equations satisfied by surface densities defined on the
    scattering obstacle. Various techniques are known to approximate the solution of those
    equations numerically. However, the system matrix of the discrete problem may be
    ill-conditioned, which regularly happens for high-resolution discretizations and 
    very low or high frequencies. 

    We aim at designing and analyzing rigourously a range new numerical schemes that
    yields well-conditions linear systems, are able to deal with complex geometry of the
    scatterers, and are robust in the low- and high-frequency limits. We focus on thin
    scatterers modeled as open curves or surfaces, so-called screens. Moreover, the new
    methods will be systematically implemented and tested numerically to illustrate the
    theoretical findings.

    First, we will derive combined-field integral equations for multi-screens, which are
    among the most general and complex types of scatterers faced in
    applications. Combined-field methods have been very successful for massive scatterers
    but are not yet available for screens. Yet, the modern quotient-space treatment of
    multi-screens seems perfectly suited for the application of this idea.

    Second, we will develop ``conformal preconditioners" for integral equations on
    polygons. At the the heart of the idea is a simple identity relating the conformal
    mapping of a given polygon into a circle to the Dirichlet-Neumann operator. This leads
    to the formulation of new preconditioners and hints at a new kind of theoretical tools
    to analyze integral equations on polygons.

    Promising partial theoretical and numerical results for acoustics in two dimensions
    are already available. The ideas naturally generalize to three dimension and, with
    suitable adaptations to electromagnetic scattering. Once elaborated for these
    problems, the full potential of our new methods will emerge and, we believe, they will
    have an impact in computational engineering. 
    }}

\section{Project details}
\drop{(rationale for the proposed work, goals, experimental approach and/or methods, milestones, expected output, specific role of the ETH Fellow, timetable; not to exceed 5 pages)}

\subsection{Topic, state of research and open problems}

\subsubsection*{Forward scattering problems in frequency domain}	

Assume that an incident time-harmonic wave with frequency $k$ impinges on an obstacle, or
a material heterogeneity, an airplane in the sky, a crack in a material, a tumor in the
body, etc.. The presence of the obstacle generates a scattered field propagating away
it. The shape and intensity of this scattered field depends on the location, geometry and
material properties of the scatterer. Scattering problems arise in different areas and
different types of waves, like sound waves (acoustics), electromagnetic waves, and elastic
waves, of which the first two will be in the focus of the project. 

In many contexts it is fundamental to predict the scattered field from the \textit{a
  priori} knowledge of the scattering object and incident wave. This can only be done by
numerical computations, and one wants high accuracy at moderate computational cost. This
is important for engineering (acoustics, electromagnetism) and is also an essential step
in most algorithms for inverse problems.

This project aims to address on the following aspects of numerical simulation of
scattering problems:
\begin{itemize}
\item[-] We look for extensions and improvements of existing numerical methods in the
  specific case where the scatterer is a ``screen", that is, an object modeled by an
  infinitely thin layer, its geometry described as an open curve (2D) or surface
  (3D). So-called multi-screens will also be considered, which are generalizations of
  screens, in particular featuring junctions.
\item[-] The key parameter is the wave frequency $k$. High-frequency problems are
  challenging because they require finer discretization. Moreover, methods that are
  efficient at medium frequencies are often haunted by instability in the high or low
  frequency ranges. In this project, we aim at designing methods with provably robust
  behavior with respect to frequency.
\end{itemize}

\subsubsection*{Integral equations and boundary element methods (BEM)}

Among the standard mathematical tools most suitable for this task are integral equation
formulations and the Galerkin boundary element method (BEM), see
\cite{mclean2000strongly,sauter2010boundary}). The discretization of integral equations by
BEM leads to dense linear systems, which, nowadays, routinely have more than $10^6$
unknowns. The use of matrix-compression techniques like Fast Multipole Methods
\cite{greengard1987fast} and H-matrices \cite{hackbusch1999sparse} becomes mandatory.
Unfortunately, this rules out the use of direct elimination solvers. The only remaining
solver option are iterative methods such as GMRES \cite{saad1986gmres}.

For Galerkin BEM for first-kind boundary integral equations, a comprehensive convergence
theory is available, but the arising linear systems are notoriously ill-conditioned for
high-resolution discretizations, causing the iterative solvers to progress very
slowly. Preconditioning becomes indispensable and much research has been devoted to it for
about thirty years. The most successful approach has been operator preconditioning
\cite{hiptmair2006operator2}, also known as Calder\'on preconditioning, pioneered in
\cite{steinbach1998construction} and further developed in \cite{BUC07ACB08} for
electromagnetics, and in \cite{hiptmair2020optimal} for screens. 

An alternative are second-kind boundary integral equations. They enjoy better stability with
respect to discretization parameters, and for this reason, in many instances, researchers
have tried to avoid the use of first-kind integral equations despite their
simplicity. However, second-kind boundary integral equations do not seems to exist for
screens and, thus, they are not relevant for this project. 

\subsubsection*{Recent advances and open problems}

Let us survey some recent advances in the field of integral equations, and related open problems.

\paragraph{Scattering by screens.} We call an obstacle complex when its is not smooth
domains or a at least a Lipschitz domain, a class of scatterers well understood by now
\cite{mclean2000strongly}. Issues also arise when singularities like edges or conical
points are present. Particular attention has been devoted to BEM for screens and cracks,
which provide important examples of non-Lipschitz domains, see
\cite{bruno2013high,gimperlein2019optimal,hiptmair2018closed,hiptmair2019preconditioning,hiptmair2020optimal,hiptmair2020compact,ramaciotti2017some}.

In the case of ``simple'' screens that can be viewed as parts of a boundary of a Lipschitz
domain, the formulation of well-posed integral equations in a proper mathematical
framework is well-understood \cite{stephan1984augmented,wendland1990hypersingular}, and
recent advances concern the design of more efficient numerical methods, including
high-order approximation schemes and provably ``optimal preconditioners". Additionally,
some more challenging geometries are considered, including multi-screens
\cite{claeys2013integral,claeys2020quotient} and fractal screens
\cite{chandlerWilde2017sobolev}. In those cases, the achievement was to obtain well-posed
integral equations and study their theoretical properties. For those complex geometries,
many natural questions remain open, e.g. convergence of the numerical methods, singularity
analysis and efficient preconditioning.

\paragraph{Robustness in frequency.} Although by now optimal preconditioners are available
for scattering at Lipschitz domains and screens,
\cite{steinbach1998construction,hiptmair2006operator}, it is difficult to precisely
determine the dependence on frequency of the performance of
preconditioners. S.~Chandler-Wilde, D.~Hewett, and E.~Spence in the UK have been made
progress in theoretical asymptotic analysis of the condition number of the classical integral equations without
preconditioners \cite{chandler2020high} and the use of asymptotic-numerical methods
\cite{chandlerWilde2015high}.

Besides the popular Calder\'{o}n preconditioners
\cite{christiansen2000preconditionneurs,christiansen2002preconditioner} for closed
surfaces (recently extended to screens \cite{bruno2012second}), ``analytic
preconditioners", based on pseudo-differential calculus which exhibit good frequency
robustness \cite{antoine2007generalized} have been proposed for closed surfaces, which
entertain a close link with regularized combined-field integral equations
\cite{buffa2005regularized}. Although the frequency robustness properties are clear in
practice and thoroughly documented in \cite{boubendir2014well}, up to now there is no
mathematical theory adapted to the study of this asymptotic robustness. Moreover, those
analytic preconditioning methods fail for screens. In the expert community there is a
belief that pseudo-differential analysis, which is often involved in constructing those
robust preconditioners, should also pave the way for a quantitative estimate of frequency
dependence, but such an desirable analysis has not been accomplished yet. Some recent
activity centering around semi-classical analysis seems promising
\cite{galkowski2019wavenumber}.

\subsection{Own (preparetory) work}

During my PhD thesis, I have contributed to the research on compression and acceleration
methods for the linear systems coming from boundary element methods
\cite{averseng2017fast}. I then worked on new \emph{high-frequency preconditioners} for acoustics
integral equations on screens in two dimensions, following the paradigm of operator
preconditioning. The first publication on this topic \cite{alouges2019new} presents the
analytical framework and some numerical tests. The new preconditioners are inspired by those of
\cite{antoine2007generalized}. For example, for the single-layer integral equation, the
following preconditioning operator is introduced:
\begin{equation}
\label{prec}
P_k = \sqrt{-(\omega \partial_\tau)^2 - k^2 \omega^2}\;.
\end{equation}
Compare this to the preconditioners of \cite{antoine2007generalized} which take the
form \[Q_k = \sqrt{-\Delta_S - k^2 I_d}\] where $\Delta_S$ is the Laplace-Beltrami
operator on the closed surface $S$.  In \eqref{prec}, $\omega$ is a weight function
defined on the screen $\Gamma$ and proportional to $\sqrt{\textup{d}(x,\partial \Gamma)}$,
while $\partial_\tau$ is the tangential derivative.  Upon using Padé approximations for
the square root, the preconditioner obtained is ``quasi-local" in the sense that it
involves only sparse matrices, as opposed to Calder\'{o}n preconditioners which are dense
and require an additional compression step. The numerical tests demonstrate the robustness
of the method with respect to $k$ and the relevance of the singular weight function
$\omega$.

Further, \cite{averseng2019pseudo} is a more in-depth theoretical paper in which I develop
a new type of \emph{pseudo-differential calculus} on open curves, providing support for the
definition of $P_k$. Indeed, it is proven that the term $k^2 \omega^2$ is the optimal
correction in terms of some pseudo-differential properties of the preconditioner. As in
the case of closed surfaces, there remains to confirm the relevance of this
pseudo-differential property in terms of numerical performance.



\subsection{Goals and methods}

\subsubsection*{Frequency-robust preconditioners on 3D screens}

Although the numerical method published in \cite{alouges2019new} is only adapted to
two-dimensional contexts, generalizations to three dimensions seem possible and has been
attempted in the last chapter of my thesis, with promising empiric results. For instance,
on an open surface $\Gamma$ with the tangential gradient $\nabla$, the preconditioner
\eqref{prec} becomes
\[P_k = \sqrt{-(\omega\,\nabla \,\omega) \cdot \nabla - k^2 \omega^2}\,.\] Once this
theory is completed for acoustic problems, a natural extension to electromagnetic
scattering will be possible, as a natural combination of my work and that of my host
professor \cite{hiptmair2019preconditioning}. Indeed, in
\cite{hiptmair2019preconditioning}, from the knowledge of \textit{closed-form inverses} of
the single layer and hypersingular operators, a compact perturbation of the inverse of the
Electric Field Integral Equation (EFIE) operator is constructed, which leads to uniform
bounds for the condition number in the medium and low frequency domain. Replacing the
exact inverses by the square-root operators (which are compact-equivalent to the exact
inverses), it is expected to obtain similar results, while benefiting from the quasi-local
form and the high-frequency robustness.

After dealing with electromagnetic problems, one can attempt to generalize this method to
other operators (Hodge-Helmholtz, Dirac) )and related scattering problems, which are
actively investigated in my host group.  Another natural research direction is the extension of the
pseudo-differential calculus of \cite{averseng2019pseudo} to open surfaces in 3D.

\subsubsection*{Conformal preconditioners}

It is widely known that in generalized combined-field formulations, the exterior Dirichlet
to Neumann (DtN) map is the optimal coupling operator. For a simply connected domain
$\Omega$ in dimension $2$, and for the Laplace problem ($k = 0$) it is possible to show
that the exterior DtN $\Lambda_0$ is given by
\[\Lambda_0 = \sqrt{-(\omega \partial_\tau)^2}\;,\]
where $\partial_\tau$ is the tangential derivative on the boundary of the polygon
and \[\omega(x) = \lvert (\nabla f) \circ f^{-1}(x)\rvert\] where $f$ is the conformal
mapping that transforms the exterior of $\Omega$ to the exterior of the unit disk. This
generalizes naturally the ideas of my thesis to more general domains in dimension 2. When
$\Omega$ is the interior of a polygon, it is possible to compute efficiently the conformal
mapping using the Schwarz-Christoffel formula \cite{driscoll2002schwarz}. By
analogy with the results of my thesis, one is led to believe that, for the Helmholtz
problem ($k \neq 0$), the operator
\[\sqrt{-(\omega \partial_\tau)^2 - k^2\omega^2}\]
may be a good approximation of the exterior DtN map $\Lambda_k$. In addition, it might be
possible to transport the pseudo-differential \rh{What is this?} to the polygon via the conformal mapping,
which would be a way to investigate this question. Preliminary theoretical and numerical
investigation are very promising.

This work would also point to ways to attach 3D singularities such as wedges, which are
very common  in practical applications.

\subsubsection*{Quotient space framework for integral equations on multi-screens}


Another promising idea is the possibility of formulating ``combined-field" integral
equations for multi-screens. Combined field formulations are well-known for smooth
scatterers and have been introduced more than 50 years ago
\cite{brakhage1965dirichletsehe} as a way to guarantee the well-posedness of the equation
at any frequency. For screens, combined field integral equations have been elusive,
because they degenerate when the obstacle no longer has interior.  However, we have
recently discovered that the formalism of multi-screens restores a ``virtual" interior to
the obstacle, and a combined-field equation can be formulated mathematically without
difficulty. Preliminary numerical simulations have shown the power of this idea and it
will be an immediate objective of this project to implement this idea and analyze it
rigorously.  \rh{Motivation: Screen first-kind BIE are well-posed for all frequencies!}

Lastly, after my thesis, I have done a 6-month post-doctorate with Xavier Claeys. At LJLL,
UPMC Paris, a former postdoc of my host professor R.~Hiptmair, on the topic of preconditioning integral
equations on multi-screens using an additive Schwarz approach, see e.g. \cite{marchand2019two}.

\subsubsection*{Implementation and numerical testing}

All the methods that will be developed in this project will be implemented and tested
numerically. For BEM simulations in dimension 2, I have developed my own code during my
thesis that I can reuse during my stay at ETH. For 3D simulations, I will either use the
software Gipsylab, a matlab BEM software which is close in implementation to my code, or
use another free software developed by the community like X.~Claeys'
Bemtools\footnote{\url{https://github.com/xclaeys/BemTool}}, already used in the host group
or BEM++\footnote{\url{http://bempp.com/}} by T.~Betcke, UCL.

\subsection{Timetable and research plan}

A crude timetable of the research project can be proposed as follows:
\begin{itemize}\itemsep-0.2em 
\item[-] High-frequency preconditioners for 3D screens for scalar waves by December 2020.
\item[-] Robust preconditioners for 3D screens for electromagnetic waves by March 2021,
  collaboration with Dr. Urzua-Torres.
\item[-] Conformal preconditioners by July 2021, collaboration with Pr. Alouges. 
\item[-] Combined-field multi-screens formulations by end of 2021, collaboration with Dr. Claeys. 
\item[-] Additive Schwarz preconditioners before end of 2021, collaboration with Dr. Claeys.
\end{itemize}

\begin{small}
  \begin{thebibliography}{0}
		
		\bibitem{alouges2019new}
		F.~Alouges and M.~Averseng.
		\newblock New preconditioners for Laplace and Helmholtz integral equations on open curves: Analytical framework and Numerical results. 
		\newblock Accepted for publication in {\em Numerische Mathematik}, 2020.

                \bibitem{ACB08}
{\sc F.~Andriulli, K.~Cools, H.~Bagci, F.~Olyslager, A.~Buffa, S.~Christiansen,
  and E.~Michielssen}, {\em A multiplicative {C}alderon preconditioner for the
  electric field integral equation}, IEEE Trans. Antennas and Propagation, 56
  (2008), pp.~2398--2412.
                
		\bibitem{antoine2007generalized}
		X.~Antoine and M.~Darbas.
		\newblock Generalized combined field integral equations for the iterative
		solution of the three-dimensional helmholtz equation.
		\newblock {\em ESAIM: Mathematical Modelling and Numerical Analysis}
		41(1):147--167, 2007.
		
		\bibitem{averseng2017fast}
		M.~Averseng.
		\newblock Fast discrete convolution in $\mathbb {R}^{2} $ with radial kernels using non-uniform fast Fourier transform with nonequispaced frequencies. 
		\newblock {\em Numerical Algorithms} 83(1):33--56, 2020.
		
		\bibitem{averseng2019pseudo}
		M.~Averseng. 
		\newblock Pseudo-differential analysis of the Helmholtz layer potentials on open curves. 
		\newblock {\em arXiv} preprint:1905.13604.
		
		\bibitem{boubendir2014well}
		Y.~Boubendir and C.~Turc.
		\newblock Well-conditioned boundary integral equation formulations for the solution of high-frequency electromagnetic scattering problems.
		\newblock {\em Computers and Mathematics with Applications} 67(10) 1772--1805, 2014.
		
		\bibitem{brakhage1965dirichletsehe}
		A.~Brakhage, P.~Werner.
		\newblock Uber das dirichletsehe aussenraumproblem fur die Helmholtzsche Schwingungsgleichung
		\newblock {\em Archive fur Mathematik} 16:325–329, 1965.
		
		\bibitem{bruno2012second}
		O.~P.~Bruno and S.~K.~Lintner.
		\newblock Second-kind integral solvers for TE and TM problems of diffraction by open arcs.
		\newblock {\em Radio Science} 47(6):1--13, 2012.
		
		\bibitem{bruno2013high}
		O.~P.~Bruno and S.~K.~Lintner.
		A high-order integral solver for scalar problems of diffraction by screens and apertures in three-dimensional space.
		\newblock {\em Journal of Computational Physics} 252:250--274, 2013.

                \bibitem{BUC07}
                  {\sc A.~Buffa and S.~Christiansen}, {\em A dual finite element complex on the
                    barycentric refinement}, Math. Comp., 76 (2007), pp.~1743--1769.

                
		\bibitem{buffa2005regularized}
		A.~Buffa and R.~Hiptmair.
		\newblock Regularized combined field integral equations.
		\newblock {\em Numerische Mathematik} 100(1):1--19, 2005.	
		
		\bibitem{christiansen2000preconditionneurs}
		S.~H.~Christiansen and J.~C.~Nédélec.
		\newblock Des préconditionneurs pour la résolution numérique des équations intégrales de frontière de l'acoustique.
		\newblock {\em Comptes Rendus de l'Académie des Sciences-Series I-Mathematics}  330(7):617--622, 2000.
		
		\bibitem{christiansen2002preconditioner}
		S.~H.~Christiansen and J.~C.~Nédélec.
		\newblock A preconditioner for the electric field integral equation based on Calderon formulas.
		\newblock {\em SIAM journal on numerical analysis}  40(3):1100--1135, 2002.
		
		\bibitem{chandlerWilde2015high}
		S.~N.~Chandler-Wilde, S.~N.~Hewett, D.~P.~ Langdon and A.~Twigger.
		\newblock A high frequency boundary element method for scattering by a class of nonconvex obstacles.
		\newblock {\em Numerische Mathematik} 129(4):647--689, 2015.
		
		\bibitem{chandlerWilde2017sobolev}
		S.~N.~ Chandler-Wilde, D.~P.~ Hewett and A.~Moiola.
		\newblock Sobolev Spaces on Non-Lipschitz Subsets of ${\mathbb {R}}^n$ with Application to Boundary Integral Equations on Fractal Screens. 
		\newblock {\em Integral Equations and Operator Theory} 87(2):179--224, 2017.
		
		\bibitem{chandler2020high}
		S.~N.~ Chandler-Wilde, E.~Spence, A.~Gibbs and V.~P.~ Smyshlyaev.
		\newblock High-frequency bounds for the Helmholtz equation under parabolic trapping and applications in numerical analysis.
		\newblock {\em SIAM Journal on Mathematical Analysis}52(1):845--893, 2020.	
		
		\bibitem{claeys2013integral}
		X.~Claeys and R.~Hiptmair.
		\newblock Integral equations on multi-screens
		\newblock {\em Integral equations and operator theory} 77(2): 167--197, 2013.
		
		\bibitem{claeys2020quotient}
		X.~Claeys, L.~Giacomel, R. Hiptmair and C.~Urzua-Torres.
		\newblock Quotient-Space Boundary Element Methods for Scattering at Complex Screens.
		\newblock {\em SAM Research Report} 2020, ETH Zurich.
		
		\bibitem{driscoll2002schwarz}
		T.~A.~Driscoll and L.~N.~Trefethen.
		\newblock {\em Schwarz-christoffel mapping}
		\newblock Cambridge University Press, 2002.
		
		\bibitem{galkowski2019wavenumber}
		J.~Galkowski and E.~A.~Spence
		\newblock Wavenumber-explicit regularity estimates on the acoustic single-and double-layer operators
		\newblock {\em Integral Equations and Operator Theory}91(6), 2019.
		
		\bibitem{gimperlein2019optimal}
		H.~Gimperlein, J.~Stocek and C. Urzua-Torres.
		\newblock Optimal operator preconditioning for pseudodifferential boundary problems.
		\newblock {\em arXiv} preprint 1905.03846, 2019.
		
		\bibitem{greengard1987fast}
		Greengard, L., Rokhlin, V.:
		\newblock A fast algorithm for particle simulations
		\newblock {\em Journal of computational physics} 73(2):325--348, 1987.
		
		\bibitem{hackbusch1999sparse}
		W.~Hackbusch.
		\newblock A sparse matrix arithmetic based on H-matrices. Part I: Introduction to H-matrices.
		\newblock {\em Computing} 62(2):89--108, 1999.
		
		\bibitem{hiptmair2006operator}
		R.~Hiptmair.
		\newblock Operator preconditioning.
		\newblock {\em Computers and mathematics with Applications} 52(5):699--706, 2006.
		
		
		\bibitem{hiptmair2018closed}
		R.~Hiptmair, C.~Jerez-Hanckes and C. Urzua-Torres.
		\newblock Closed-form inverses of the weakly singular and hypersingular operators on disks.
		\newblock {\em Integral Equations and Operator Theory} 90(1), 2018.
		
		
		\bibitem{hiptmair2019preconditioning}
		R.~Hiptmair and C.~Urzua Torres.
		Preconditioning the EFIE on Screens
		\newblock {\em SAM Research Report} 2019, ETH Zurich.
		
		
		\bibitem{hiptmair2020optimal}
		R.~Hiptmair, C.~Jerez-Hanckes and C.~Urzua-Torres.
		\newblock Optimal operator preconditioning for Galerkin boundary element methods on 3D screens. 
		\newblock {\em SIAM J. Numer. Anal.} 58(1):834--857, 2020.
		
		\bibitem{hiptmair2020compact}
		R.~Hiptmair and C.~Urzua-Torres.
		\newblock Compact equivalent inverse of the {E}lectric {F}ield
		{I}ntegral {O}perator on screens.
		\newblock {\em Integral Equations and Operator Theory} 92(1), 2020.

               
		\bibitem{marchand2019two}
		P.~Marchand, P.~Jolivet, P.~H.~Tournier, X.~Claeys and F.~Nataf
		\newblock Two-level preconditioning for h-version boundary element approximation of hypersingular operator with GenEO.
		\newblock {Archives ouvertes} preprint hal-0218871, 2019.

		
		\bibitem{mclean2000strongly}
		W.~C.~H. McLean.
		\newblock {\em Strongly elliptic systems and boundary integral equations}.
		\newblock Cambridge university press, 2000.
		
		\bibitem{ramaciotti2017some}
		J.~C.~Nédélec and P.~Ramaciotti.
		\newblock About some boundary integral operators on the unit disk related to the Laplace equation
		\newblock {\em SIAM Journal on Numerical Analysis} 55(4):1892--1914, 2017.
		
		
		\bibitem{saad1986gmres}
		Y.~Saad and M.~H. Schultz.
		\newblock GMRES: A generalized minimal residual algorithm for solving
		nonsymmetric linear systems.
		\newblock {\em SIAM Journal on scientific and statistical computing},
		7(3):856--869, 1986.
		
		\bibitem{sauter2010boundary}
		S.~Sauter and C.~Schwab.
		\newblock {\em Boundary Element Methods}
		\newblock Springer 2010.
		
		\bibitem{stephan1984augmented}
		E.~P.~Stephan and W.~L.~Wendland.
		\newblock An augmented Galerkin procedure for the boundary integral method applied to two-dimensional screen and crack problems
		\newblock {\em Applicable Analysis} 18(3):183--219, 1984.
		
		
		\bibitem{steinbach1998construction}
		O.~Steinbach and W.~L. Wendland.
		\newblock The construction of some efficient preconditioners in the boundary
		element method.
		\newblock {\em Advances in Computational Mathematics} 9(1-2):191--216, 1998.
		
		\bibitem{wendland1990hypersingular}
		\newblock A hypersingular boundary integral method for two-dimensional screen and crack problems.
		\newblock {\em Archive for rational mechanics and analysis} 112(4):363--390, 1990.
		

		
		
	\end{thebibliography}
\end{small}

\section*{2.3 Available resources of the ETH host professor for the realization of the project}
(Equipment, staff, knowledge, experience)\\

The proposed project is of a theoretical and computational nature without an experimental component and does not require any special equipment of facilities, beyond standard resources like office space and IT infrastructure. Researchers a the Seminar of Applied Mathematics have access to D-MATH compute servers and ETH's Euler and Leonhard Linux clusters, which can be used for large-scale computations to be carried out as part of the project.

\section*{2.4 Significance of the project for the ETH host professor} 

Prof. R. Hiptmair's research has had a focus on boundary integral equations, boundary element methods, and preconditioning techniques for many years. In particular he has worked on operator preconditioning, screen problems, and novel BIE approaches to electromagnetic scattering. The natural continuation of this work are investigations into operator preconditioning for complex screens and the developmen of preconditioning strategies that are robust for high and low (in the case of electromagnetics) frequencies. Thus Prof. Hiptmair's research interests are perfectly aligned with the goals of the fellowship application.


\section*{2.5 Planned use of requested budget for research costs }
(including travel costs, conference attendances, consumables)\\

Beside kCHF 4 to be spent for travel, I am planning to use the remaining kCHF 8 of the standard budget of kCHF 12 to pay the salary of 1-2 student research assistants, who are to help me with coding jobs. I hope to be able to hire students after they have done term or thesis projects with me. Thus they can make a substantial contribution to software development in the project.

\section*{2.6 Proposed teaching duties of ETH Fellowship candidate}
(including an estimate of the average load)\\

Upon arriving at ETH Zurich I would like to teach an MSc-level course on "Integral
Equation Methods for Scattering" beside a Seminar on "Operator Preconditioning". I hope to attract several undergraduate students interested in doing MSc and BSc thesis or term projects under my supervision, thus building a small informal group. I expect a ~30\% workload due to teaching proper (~15\%) and supervision (~15\%), though the latter will be closely connected to my research.

\section*{2.7 Other requests for funding, especially applications submitted to other fellowship programs}
	(Have you applied to other funding bodies with a similar or the same research project? Please note that you should also inform us if you obtain or request additional funds from other funding bodies during the evaluation period of your application.\\
	
I have not submitted any other applications.

\section*{2.8 List of relevant publications by the ETH Fellow candidate and the ETH host professor}

\begin{thebibliography}{0}
	% Numbers 2 to avoid bibliographic collisions.
	\bibitem{alouges2019new2}
	F.~Alouges and M.~Averseng.
	\newblock New preconditioners for Laplace and Helmholtz integral equations on open curves: Analytical framework and Numerical results. 
	\newblock Accepted for publication in {\em Numerische Mathematik}, 2020.
	
	\bibitem{averseng2017fast2}
	M.~Averseng.
	\newblock Fast discrete convolution in $\mathbb {R}^{2} $ with radial kernels using non-uniform fast Fourier transform with nonequispaced frequencies. 
	\newblock {\em Numerical Algorithms} 83(1):33--56, 2020.
	
	\bibitem{averseng2019pseudo2}
	M.~Averseng. 
	\newblock Pseudo-differential analysis of the Helmholtz layer potentials on open curves. 
	\newblock {\em arXiv} preprint:1905.13604.
	
	\bibitem{CLH17}
	X.~Claeys and R.~Hiptmair.
	\newblock First-kind boundary integral equations for the {H}odge-{H}elmholtz operator.
	\newblock {\em SIAM Journal on Mathematical Analysis} 51(1):197--227, 2019.
	
	\bibitem{CLH18t}
	X.~Claeys and R.~Hiptmair.
	\newblock First-Kind Galerkin Boundary Element Methods for the {H}odge-{L}aplacian in Three Dimensions.
	\newblock {\em SAM report} 2018(35), ETH Zurich.
	
	\bibitem{claeys2020quotient2}
	X.~Claeys, L.~Giacomel, R. Hiptmair and C.~Urzua-Torres.
	\newblock Quotient-Space Boundary Element Methods for Scattering at Complex Screens.
	\newblock {\em SAM Research Report} 2020, ETH Zurich.
	
	
	\bibitem{hiptmair2006operator2}
	R.~Hiptmair.
	\newblock Operator preconditioning.
	\newblock {\em Computers and mathematics with Applications} 52(5):699--706, 2006.
	
	
	\bibitem{hiptmair2018closed2}
	R.~Hiptmair, C.~Jerez-Hanckes and C. Urzua-Torres.
	\newblock Closed-form inverses of the weakly singular and hypersingular operators on disks.
	\newblock {\em Integral Equations and Operator Theory} 90(1), 2018.
	
	\bibitem{hiptmair2019preconditioning2}
	R.~Hiptmair and C.~Urzua Torres.
	Preconditioning the EFIE on Screens
	\newblock {\em SAM Research Report} 2019, ETH Zurich.
	

	\bibitem{hiptmair2020optimal2}
	R.~Hiptmair, C.~Jerez-Hanckes and C.~Urzua-Torres.
	\newblock Optimal operator preconditioning for Galerkin boundary element methods on 3D screens. 
	\newblock {\em SIAM J. Numer. Anal.} 58(1):834--857, 2020.
	
	\bibitem{hiptmair2020compact2}
	R.~Hiptmair and C.~Urzua-Torres.
	\newblock Compact equivalent inverse of the {E}lectric {F}ield
	{I}ntegral {O}perator on screens.
	\newblock {\em Integral Equations and Operator Theory} 92(1), 2020.
	
\end{thebibliography}


\section*{2.9 Relevant publications of other authors}


\begin{thebibliography}{0}
	\bibitem{antoine2007generalized2}
	X.~Antoine and M.~Darbas.
	\newblock Generalized combined field integral equations for the iterative
	solution of the three-dimensional helmholtz equation.
	\newblock {\em ESAIM: Mathematical Modelling and Numerical Analysis}
	41(1):147--167, 2007.
	
	\bibitem{bruno2012second2}
	O.~P.~Bruno and S.~K.~Lintner.
	\newblock Second-kind integral solvers for TE and TM problems of diffraction by open arcs.
	\newblock {\em Radio Science} 47(6):1--13, 2012.
	
	\bibitem{galkowski2019wavenumber2}
	J.~Galkowski and E.~A.~Spence
	\newblock Wavenumber-explicit regularity estimates on the acoustic single-and double-layer operators
	\newblock {\em Integral Equations and Operator Theory}91(6), 2019.
	
	\bibitem{gimperlein2019optimal2}
	H.~Gimperlein, J.~Stocek and C. Urzua-Torres.
	\newblock Optimal operator preconditioning for pseudodifferential boundary problems.
	\newblock {\em arXiv} preprint 1905.03846, 2019.
	
	\bibitem{steinbach1998construction2}
	O.~Steinbach and W.~L. Wendland.
	\newblock The construction of some efficient preconditioners in the boundary
	element method.
	\newblock {\em Advances in Computational Mathematics} 9(1-2):191--216, 1998.	
\end{thebibliography}


\end{document}
