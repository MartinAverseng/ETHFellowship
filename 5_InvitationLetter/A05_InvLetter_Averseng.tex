%\documentclass[ngerman]{ethbrief3}
\documentclass[english]{ethbrief3}
% \documentclass[ngerman,DepLogo]{ethbrief3}
%\documentclass[english,DepLogo]{ethbrief3}
%\documentclass[ngerman,ETHUZH]{ethbrief3}
%\documentclass[english,ETHUZH]{ethbrief3}

\usepackage[utf8]{inputenc}
\usepackage[pdfborder={0 0 0},pdftex]{hyperref}
\usepackage{graphicx}
\usepackage{amsmath}
\usepackage{amssymb}
\usepackage{color}

% automatic language support (Niklas Beisert)
\EthLang{
  \RequirePackage[document]{ragged2e}% Corrects hyphenation in german
  \RequirePackage{ngerman}
  \PassOptionsToPackage{ngerman}{babel}
}{
  \PassOptionsToPackage{english}{babel}
}

% Declaration of the personal data
    \EthFirstUnitName{\EthLang{SAM, D-MATH}{SAM, D-MATH}}
    % \EthSecondUnitName{\EthLang{}{2nd line Organisation unit}}
    \EthName{ETH \Zurich}
    \EthPersName{\EthLang{Dr.~Ralf Hiptmair}{Dr.~Ralf Hiptmair}}
    \EthPersPosition{\EthLang{Professor}{Professor}}
     \EthOffice{\EthLang{HG G 58.2}{HG G 58.2}}
     \EthStreet{\EthLang{R\"amistrasse 101}{R\"amistrasse 101}}
     \EthTown{8092 \Zurich}
%     \EthAddress{\EthLang{Freie\\Adresseingabe}{freestyle\\address}}
% --- country optional, typical if sendig outside CH
%     \EthLang{}{\EthCountry{\Switzerland}} % only the country in english version
     \EthCountry{\Switzerland} % in german and english version
     \EthPhone {+41 44 632 3404}
%     \EthPhone {+41 44 632 bb bb} % secondary phone, fax, \ldots
     \EthFax   {+41 44 632 1104}
%     \EthMobile{+41 7 59 aa bbb}
\EthLang{\EthEmail{hiptmair@sam.math.ethz.ch}
     % \EthWeb{www.max.muster.ethz.ch}
}{
     % \EthEmail{john.doe@ethz.ch}
  \EthWeb{www.sam.math.ch/\symbol{126}hiptmair}
}
%     \EthPersInfo{\EthLang{Zus\"atzliche\\Daten}{additional\\data}}

% end of declaration of the personal data

%***** Einbinden einer Unterschift-Grafik
\signaturefile[0.5cm]{\includegraphics[height=1cm]{signature.pdf}} % include a signature file
 \signature{Ralf Hiptmair}

%\makelabels         % to get the address for an envelope
% \date{May 15, 2013} % be free to define the date manually
%***** multilinguale Ortsnamen, \Zurich definiert als Standard
% \location{\Zurich}  % be free to define the location manually

%***** Umbenannt
\subject[{\color{white}response}]{Concern: Invitation Letter for Dr. Martin Averseng,
  Applicant for ETH Zurich Postdoctoral Fellowship}

%***** Umgehung von \begin{letter} moeglich, Angabe der Zieladresse
\recipient{
  Prof. Dr. Detlef Günther \\
  Vice President for Research \\
  ETH Zurich \\
  \texttt{ethfellows@sl.ethz.ch}
}

\begin{document}

%***** Umgehung von \begin{letter} moeglich
%  \begin{letter}{Adresse}

\opening{Sehr geehrter Herr Kollege Günther,}

with this letter I would like to justify my \textbf{strong support} for and interest in
the ETH Zurich Postdoctoral Fellowship Application. I had not known Dr. Averseng until
last year Prof.~Francois Alouges from \'Ecole Polytechnique, Paris, asked me whether I
would be wiling to serve as a reviewer for his doctoral thesis on "Méthodes efficaces pour
la diffraction acoustique en 2 et 3 dimensions: Préconditionnement sur des domaines
singuliers et convolution rapide". After browsing the related research reports, I agreed,
because I realized the fundamental nature and substantial potential of that thesis'
results, and their importance for my own work.

Dr.~Averseng's PhD research has pushed the frontier of numerical and mathematical analysis
of boundary integral equations (BIEs) for scattering at open curves and surfaces in
several directions:
\begin{itemize}
\item He was the first to establish a framework of scales of weighted function spaces,
  which turns out to be \emph{the} right setting for understanding the BIEs.
\item He pioneered the development of a pseudo-differential calculus for \emph{open}
  curves.
\item He exploited the new pseudo-differential calculus for the construction of 
  micro-local preconditioners for Helmholtz BIEs. As demonstrated by numerical tests those
  allow efficient implementation and perform excellently over wide ranges of frequencies.
\end{itemize}
This work provides many promising starting points for further research, which can tackle open
surfaces in 3D (preliminary work already documented in the thesis) and scattering problems
in electromagnetics and linear elasticity.

The focus and direction of Dr.~Averseng's research \textbf{perfectly complements} my own
work of the past few years. I have maintained a strong interest in BIE methods for
scattering for many years, first confined to closed surfaces and transmission problems,
but shifting to open curves in 2014, when I won an ETHIRA grant for a project about
"Scattering at complex screens". Jointly with C.~Urz\'ua-Torres in her PhD work we
succeeded in finding operator preconditioners for Galerkin boundary element methods (BEM)
for all first-kind BIEs for both acoustic and electromagnetics. However, our work has a
blind spot: we could never make explicit the dependence of crucial constants on the
frequency. Further progress will require techniques like those invented by Dr.~Averseng.

Since 2014 I also collaborated with Dr.~X.~Claeys from LJLL, UPMC, Paris, to achieve a
deeper understanding of BIEs on complicated open surfaces, so-called \emph{complex
  screens}. We managed to introduce a quotient-space framework and also harness it for
novel boundary element discretizations in $[${\sc X.~Claeys, L.~Giacomel, R.~Hiptmair, and
  C.~Urzua-Torres}, {\em Quotient-space boundary element methods for scattering at complex
  screens}, Tech. Rep. 2020-11, Seminar for Applied Mathematics, ETH Z{\"u}rich, under
review$]$. In recent, still unpublished work Dr.~Averseng could exploit this idea to
construct novel BIEs for screen problems. His ideas also seem to be highly relevant for
building frequency-robust preconditioners for complex screens. All this is closely
connected to my ongoing SNF-funded project dedicated to ``Novel Boundary Element Methods
for Electromagnetics'', on which two PhD students are employed currently. Thus from the
very beginning Dr. Averseng will be embedded in a group, whose members share almost all of
his research interests.

Hardly surprising, I have already offered Dr.~Averseng a full postdoctoral position in my
group and he will start this week on September 1, 2020. I should add that Dr.~Averseng
prevailed in a competitive postdoc application procedure implemented on the level of the
institute, the Seminar for Applied Mathematics. His research achievements and presentation
skills made him the first choice.

The ETH fellowship would be another recognition of his prowess as a researcher in applied
mathematics. It would enhance his profile in the community and give him greater
independence in his contributions to teaching. His outstanding communication skills would also
make him a great messenger for ETH as a leading research institution.

Summing up, Dr. Averseng is a \textbf{worthy and outstanding candidate} for an ETH
Postdoctoral Fellowship and, in my opinion the case for selecting him is very strong.
	
   \closing{Yours sincerely,}

    % \ps{Postscriptum}
    % \encl{Anhang}
    % \cc{carbon copy}
  % \end{letter}

\end{document}