%%%%%%%%%%%%%%%%%%%%%%%%%%%%%%%%%%%%%%%%%
% Medium Length Professional CV
% LaTeX Template
% Version 2.0 (8/5/13)
%
% This template has been downloaded from:
% http://www.LaTeXTemplates.com
%
% Original author:
% Rishi Shah 
%
% Important note:
% This template requires the resume.cls file to be in the same directory as the
% .tex file. The resume.cls file provides the resume style used for structuring the
% document.
%
%%%%%%%%%%%%%%%%%%%%%%%%%%%%%%%%%%%%%%%%%

%----------------------------------------------------------------------------------------
%	PACKAGES AND OTHER DOCUMENT CONFIGURATIONS
%----------------------------------------------------------------------------------------

\documentclass{resume} % Use the custom resume.cls style

\usepackage[left=0.75in,top=0.6in,right=0.75in,bottom=0.6in]{geometry} % Document margins
\newcommand{\tab}[1]{\hspace{.2667\textwidth}\rlap{#1}}
\newcommand{\itab}[1]{\hspace{0em}\rlap{#1}}
\name{MARTIN AVERSENG} % Your name
\address{20 rue de l'Armorique, 75015 Paris}
\address{(+33)622854025 \\ martin.averseng@gmail.com} % Your phone number and email
\usepackage[utf8]{inputenc}
\begin{document}

%----------------------------------------------------------------------------------------
%	EDUCATION SECTION
%----------------------------------------------------------------------------------------

\begin{rSection}{Education}
{\bf CMAP, Ecole Polytechnique, Palaiseau} \hfill {\em Sept. 2016 - Dec. 2019} 
\\ PhD thesis in applied mathematics:\\
{\em Efficient methods in acoustic scattering in 2D and 3D\\
Preconditioning on singular domains and fast convolution.} \hfill Direction: Pr. François Alouges. \\
Thesis defended and obtained on october 14th 2019.\\ 
{\bf Université Pierre et Marie Curie, Paris} \hfill {\em Sept. 2015 - July 2016} 
\\ Master's degree, Numerical analysis of partial differential equations.\\
{\bf IRCAM, Paris} \hfill {\em Sept. 2014 - July 2015} 
\\ Master's degree, Acoustics, signal processing, computer science applied to music. \\
{\bf Ecole Polytechnique, Palaiseau} \hfill {\em Sept. 2011 - July 2014}\\
Major in applied mathematics. \\
Minors in quantum and statistical physics, continuum mechanics

%Minor in Linguistics \smallskip \\
%Member of Eta Kappa Nu \\
%Member of Upsilon Pi Epsilon \\


\end{rSection}

\begin{rSection}{Work experience}
{\bf Laboratoire Jacques-Louis Lions, Inria Alpines team, Paris} \hfill {\em Jan. 2020 - Present} \\
Post-doc, supervised by Xavier Claeys. Working on combinations of additive Schwarz and Calder\'{o}n preconditioners.\\
{\bf Laboratoire des systèmes perceptifs, ENS, Paris} \hfill {\em January 2015 - July 2015} 
\\ Research internship in behavioral neurosciences.\\ 
{\bf ESI-Group, San Diego}
\hfill {\em March - July 2013} \\
Research internship. \\Modeling of the variance in a transient model of the Statistical Energy Analysis. \\
{\bf PSA Peugeot Citroën, Vélizy-Villacoublay} \hfill{\em July - Sept. 2013} \\
Ergonomy of Human Machine interfaces \\
{\bf Armée française, 8ème RPIMA, Castres}  \hfill{\em Sept. 2011 - April 2012}\\
8 month experience in French army, including 4 month at the 8ème RPIMA. 
\end{rSection}

\begin{rSection}{Publications}
- Bagur, S., Averseng, M., Elgueda, D., David, S., Fritz, J., Yin, P., Shamma, S., Boubenec, Y., and Ostojic, S.: Go no-go task engagement enhances population representation of the target stimulus in primary auditory cortex. {\em Nature Comm.} 9(1), 2529 (2018)

- Averseng M.: Fast discrete convolution in R2
with radial kernels using non-uniform fast Fourier transform with nonequispaced frequencies. {\em Numer. Algor.} (2019)

- Alouges. F., Averseng, M.: New preconditioners for the Laplace and Helmholtz integral equations on open curves. Submitted. Arxiv preprints 1905.13604 (2019)

-Averseng, M.: Pseudo-differential analysis of the weighted layer potentials for the Laplace and Helmholtz integral equations on open curves. Submitted. Arxiv preprint 1905.13602 (2019).
\end{rSection}


\begin{rSection}{Language and computer skills}
{\bf Programming:}  {Matlab and Python.} \\
{\bf Languages:}  {Fluent in english and italian, intermediate level in spanish. French native speaker}
\end{rSection}

\end{document}